\hypertarget{group__stack}{
\section{Stack}
\label{group__stack}\index{Stack@{Stack}}
}
\subsection*{Data Structures}
\begin{CompactItemize}
\item 
struct \hyperlink{structstack}{stack}
\begin{CompactList}\small\item\em This is a stack struct.\item\end{CompactList}\end{CompactItemize}
\subsection*{Functions}
\begin{CompactItemize}
\item 
int \hyperlink{group__stack_a0}{push} (\hyperlink{structstack}{stack} s, int val)
\begin{CompactList}\small\item\em This function is used to push stuff.\item\end{CompactList}\item 
int \hyperlink{group__stack_a1}{pop} (\hyperlink{structstack}{stack} s)
\begin{CompactList}\small\item\em This pops.\item\end{CompactList}\item 
int \hyperlink{group__stack_a2}{top} (\hyperlink{structstack}{stack} s)
\begin{CompactList}\small\item\em Take a look at the top.\item\end{CompactList}\end{CompactItemize}


\subsection{Function Documentation}
\hypertarget{group__stack_a1}{
\index{Stack@{Stack}!pop@{pop}}
\index{pop@{pop}!Stack@{Stack}}
\subsubsection[pop]{\setlength{\rightskip}{0pt plus 5cm}int pop (\hyperlink{structstack}{stack} {\em s})}}
\label{group__stack_a1}


This pops.

It is used to pop stuff

\begin{Desc}
\item[Warning: ]\par
This can give you an error if there is nothing to pop \end{Desc}
\begin{Desc}
\item[See also: ]\par
stack , push \end{Desc}
\begin{Desc}
\item[Parameters: ]\par
\begin{description}
\item[{\em 
s}]the stack to get an element from\end{description}
\end{Desc}
\begin{Desc}
\item[Returns: ]\par
\end{Desc}
\hypertarget{group__stack_a0}{
\index{Stack@{Stack}!push@{push}}
\index{push@{push}!Stack@{Stack}}
\subsubsection[push]{\setlength{\rightskip}{0pt plus 5cm}int push (\hyperlink{structstack}{stack} {\em s}, int {\em val})}}
\label{group__stack_a0}


This function is used to push stuff.

It puts stuff on the end.

\begin{Desc}
\item[\hyperlink{bug__bug000001}{Bug: }]\par
this function rules too much\end{Desc}
 \begin{Desc}
\item[Parameters: ]\par
\begin{description}
\item[{\em 
s}]This is the stack to push onto \item[{\em 
val}]This is the value to push\end{description}
\end{Desc}
\begin{Desc}
\item[Returns: ]\par
\begin{CompactItemize}
\item 
a non-negative value indicates success\item 
a negative return value indicates that the push was not successful\item 
The - creates a bullet list \end{CompactItemize}
\end{Desc}
\begin{Desc}
\item[Examples: ]\par
\hyperlink{stack__example_8c-example}{stack\_\-example.c}.\end{Desc}


Definition at line 13 of file stack.c.

References stack::end, and stack::stack.\hypertarget{group__stack_a2}{
\index{Stack@{Stack}!top@{top}}
\index{top@{top}!Stack@{Stack}}
\subsubsection[top]{\setlength{\rightskip}{0pt plus 5cm}int top (\hyperlink{structstack}{stack} {\em s})}}
\label{group__stack_a2}


Take a look at the top.

This is another way to do documentation. It is better to put the documentation in the source, because then you can edit them together.

\begin{Desc}
\item[Returns: ]\par
The top of the stack \end{Desc}
